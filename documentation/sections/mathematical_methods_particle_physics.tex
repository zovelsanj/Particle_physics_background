\section{PARTICLE PHYSICS}
\subsection{Discrete Symmetries}

\href{https://physics.stackexchange.com/questions/178986/how-is-jpc-value-experimentally-determined-for-new-types-of-particles}{\textbf{J}$^{\textbf{PC}}$}

\subsection{Spatial Rotations}
\textbf{Clebsch-Gordan coefficients}
\href{URL}{generator of rotations}
\href{https://en.wikipedia.org/wiki/Wigner_D-matrix}{Wigner D-matrix}

Helicity conservation in $e^+ e^- \rightarrow \mu^+ \mu^-$

Helicity, $\sigma$. $\frac{p}{|p|}$, is the projection of the spin along the momentum direction.

\subsection{Lorentz Invariance}
Most high-energy physics requires energy scales E $>>$ m$_p$, so it
is essential that the requirements of special relativity be respected. In
practice, this means identifying suitable 4-vectors and Lorentz invariants. 
Although the position and direction of particles is important when
actually performing experiments, the results are most often derived from
knowledge of the energy and momentum of the interacting particles.

\subsection{Rapidity}
High-energy hadron-hadron interactions tend to produce final states
with limited transverse momentum with respect to the initial beam direction. 
In such circumstances, rapidity (y) and transverse mass (m$_T$) are convenient variables.
\begin{equation}
    y = \frac{1}{2} ln \Big(\frac{E+p_z}{E-p_z}\Big), \hspace{1cm} m_T = \sqrt{m^2 + p_T^2}
\end{equation}


\newpage